\documentclass[12pt]{article}

\usepackage{fancyhdr}
\usepackage[fleqn]{amsmath}
\usepackage{amssymb}
\usepackage{hyperref}

\pagestyle{fancy}
\fancyhf{}
\lhead{\textbf{MAU22C00} Discrete Mathematics}
\rhead{Ted Johnson ‑ 19335618}
\rfoot{\thepage}

\newcommand{\qedsymbol}{\rule{0.7em}{0.7em}}

\urlstyle{same}
\hypersetup{colorlinks=true, linkcolor=blue, urlcolor=blue}

\begin{document}

\section*{Assignment 1}

I have read and I understand the plagiarism provisions in the General Regulations of the University Calendar for the current year, found at \href{http://www.tcd.ie/calendar}{here}. I have also completed the Online Tutorial on avoiding plagiarism ‘Ready Steady Write’, located \href{http://tcd-ie.libguides.com/plagiarism/ready-steady-write}{here}.

\subsection*{Exercise 1}

Please carry out the following proof in propositional logic following the proof format in tutorial 1.\\
Hypotheses: $ P \Rightarrow (Q \Leftrightarrow \neg R),\ P \vee \neg S,\ R \Rightarrow S,\ \neg Q \Rightarrow \neg R $\\
Conclusion: $ \neg R $

\subsubsection*{Solution}

\begin{align*}
  \intertext{These are the provided hypotheses:}
    \bullet\ \ & P \Rightarrow (Q \Leftrightarrow \neg R)                  \tag*{\textcircled{a}} \label{eq:1.a} \\
    \bullet\ \ & P \vee \neg S                                             \tag*{\textcircled{b}} \label{eq:1.b} \\
    \bullet\ \ & R \Rightarrow S                                           \tag*{\textcircled{c}} \label{eq:1.c} \\
    \bullet\ \ & \neg Q \Rightarrow \neg R                                 \tag*{\textcircled{d}} \label{eq:1.d} \\
  \intertext{First, let's used tautology \#21 to construct some implications:}
    & P \lor \neg S                                                       \tag{Using\ \ref{eq:1.b}} \\
    \rightarrow\ \ & \neg S \lor P                                        \tag{\#32: Law of commutativity} \\
    \rightarrow\ \ & S \Rightarrow P                                      \tag{\#21} \\
    \rightarrow\ \ & \neg P \Rightarrow \neg S                            \tag{\#24: Law of contraposition} \\
    \bullet\ \ & \neg P \Rightarrow \neg S                                \tag*{\textcircled{1}} \label{eq:1.1} \\
    & R \Rightarrow S                                                     \tag{Using\ \ref{eq:1.c}} \\
    \rightarrow\ \ & \neg S \Rightarrow \neg R                            \tag{\#24: Law of contraposition} \\
    \bullet \ \ & \neg S \Rightarrow \neg R                               \tag*{\textcircled{2}} \label{eq:1.2} \\
    & P \Rightarrow (Q \Leftrightarrow \neg R)                            \tag{Using\ \ref{eq:1.a}} \\
    \rightarrow\ \ & \neg (Q \Leftrightarrow \neg R) \Rightarrow \neg P   \tag{\#21} \\
    \bullet\ \ & \neg (Q \Leftrightarrow \neg R) \Rightarrow \neg P       \tag*{\textcircled{3}} \label{eq:1.3} \\
  \intertext{Now we use tautology \#14 to prune the unnecessary variables:}
    & \neg P \Leftrightarrow \neg R                                       \tag{Using Biconditional Rule with\ \ref{eq:1.1} and\ \ref{eq:1.2}}\\
    \rightarrow\ \ & \neg (Q \Leftrightarrow \neg R) \Rightarrow \neg R   \tag{Substituting into\ \ref{eq:1.3}} \\
    \bullet\ \ & \neg (Q \Leftrightarrow \neg R) \Rightarrow \neg R       \tag*{\textcircled{4}} \label{eq:1.4} \\
  \intertext{Next, we are going to reorganise\ \ref{eq:1.4}:}
    & \neg (Q \Leftrightarrow \neg R) \Rightarrow \neg R                                            \tag{Using\ \ref{eq:1.4}} \\
    \rightarrow\ \ & \neg ((Q \land \neg R) \lor (\neg Q \land \neg \neg R)) \Rightarrow \neg R     \tag{\#23} \\
    \rightarrow\ \ & \neg (Q \land \neg R) \land \neg (\neg Q \land \neg \neg R) \Rightarrow \neg R \tag{\#19: De Morgan's Law} \\
    \rightarrow\ \ & (\neg Q \lor \neg \neg R) \land (\neg \neg Q \lor \neg \neg \neg R) \Rightarrow \neg R \tag{\#18: De Morgan's Law} \\
    \rightarrow\ \ & (\neg Q \lor R) \land (Q \lor \neg R) \Rightarrow \neg R                       \tag{\#3: Law of Double Negation} \\
    \rightarrow\ \ & \neg Q \lor R \Rightarrow Q \lor \neg R \Rightarrow \neg R                     \tag{\#27} \\
    \bullet\ \ & \neg Q \lor R \Rightarrow Q \lor \neg R \Rightarrow \neg R                         \tag*{\textcircled{5}} \label{eq:1.5} \\
  \intertext{Finally, we assert $\neg R$ by using \textit{Modus Ponens}:}
    & \neg Q \Rightarrow \neg R                                         \tag{Using\ \ref{eq:1.d}} \\
    \rightarrow\ \ & Q \lor \neg R                                      \tag{\#21} \\
    \rightarrow\ \ & Q \lor \neg R \Rightarrow \neg R                   \tag{\#10: Modus Ponens with\ \ref{eq:1.5}} \\
    \rightarrow\ \ & \neg R                                             \tag{\#10: Modus Ponens with\ \ref{eq:1.5}} \\
    & \neg R
\end{align*}
\null\hfill $\blacksquare$
\pagebreak

\subsection*{Exercise 2}

Prove the following statement: If $n$ is any integer, then $n^{2}-3n$ must be even.

\subsubsection*{Solution}

For $n^{2}-3n$ to be even, the expression must take the form $2k$ for any integer $n$. Here, we will prove this is true if $n$ is either even or odd.
\begin{align*}
  \intertext{ In the case that $n$ is an even number, $2k$ can be substituted for $n$: }
  & n^{2}-3n \\
  \rightarrow\ \ & {(2k)}^{2}-3(2k) \\
  \rightarrow\ \ & 4k^{2}-6k \\
  \rightarrow\ \ & 2{(2k^{2}-3k)} \\
  \intertext{ We have proven $n^{2}-3n$ is even when $n$ is even, as the expression $2{(2k^{2}-3k)}$ takes the form $2k$. }
  \intertext{ In the case that $n$ is an odd number, $2k+1$ can be substituted for $n$: }
  & n^{2}-3n \\
  \rightarrow\ \ & {(2k+1)}^{2}-3(k+1) \\
  \rightarrow\ \ & 4k^{2}+1-6k-3 \\
  \rightarrow\ \ & 4k^{2}-6k-2 \\
  \rightarrow\ \ & 2{(2k^{2}-3k-1)}
\end{align*}
We have proven $n^{2}-3n$ is even when $n$ is odd, as the expression $2{(2k^{2}-3k-1)}$ takes the form $2k$.
\\\\
$\therefore\ n^{2}-3n$ is even for any integer $n$.
\\\null\hfill $\blacksquare$
\pagebreak

\subsection*{Exercise 3}

Prove via inclusion in both directions that for any three sets $A$, $B$ and $C$:
\[ A \cap (B \setminus C) = (A \cap B) \setminus (A \cap C) \]

\subsubsection*{Solution}

To prove via double inclusion, we must prove both $A \cap (B \setminus C)$ contains $(A \cap B) \setminus (A \cap C)$ and $(A \cap B) \setminus (A \cap C)$ contains $A \cap (B \setminus C)$, or, more formally, prove both of these:
\begin{align}
  & (A \cap B) \setminus (A \cap C) \subseteq A \cap (B \setminus C) \tag*{\textcircled{a}} \label{eq:3.a} \\
  & A \cap (B \setminus C) \subseteq (A \cap B) \setminus (A \cap C) \tag*{\textcircled{b}} \label{eq:3.b}
\end{align}
\\
\noindent To prove\ \ref{eq:3.a}, take $\forall x \in (A \cap B) \setminus (A \cap C)$:
\begin{align*}
    & x \in (A \cap B) \setminus (A \cap C) \\
    \rightarrow\ \ & x \in A \cap B \cap (A^{c} \cup C^{c}) \tag{De Morgan's law} \\
    \rightarrow\ \ & x \in ((A \cap B) \cap A^{c}) \cup ((A \cap B) \cap C^{c}) \tag{Law of distributivity} \\
    \rightarrow\ \ & x \in (A \cap A^{c} \cap B) \cup (A \cap B \cap C^{c}) \tag{Law of associativity} \\
  \intertext{As $A \cap A^{c} \cap B$ is always false, we can exclude the left-hand side:}
    \rightarrow\ \ & x \in A \cap B \cap C^{c} \tag{Excluding false left-side of union} \\
    \rightarrow\ \ & x \in A \cap (B \setminus C) \tag{Applying definition of set subtraction} \\
    & x \in A \cap (B \setminus C) \tag*{Here\ \ref{eq:3.a} is proven}
\end{align*}
\\
\noindent To prove\ \ref{eq:3.b}, take $\forall x \in A \cap (B \setminus C)$:
\begin{align*}
    & x \in A \cap (B \setminus C) \\
    \rightarrow\ \ & x \in A \cap B \cap C^{c} \tag{Definition of set subtraction} \\
    \rightarrow\ \ & x \in A \cap B \cap (A^{c} \cup C^{c}) \tag{\begin{math}A \cup A^{c}\end{math} is an empty set} \\
    \rightarrow\ \ & x \in A \cap B \cap {(A \cap C)}^{c} \tag{De Morgan's law} \\
    \rightarrow\ \ & x \in (A \cap B) \setminus (A \cap C) \tag{Applying definition of set subtraction} \\
    & x \in (A \cap B) \setminus (A \cap C) \tag*{Here\ \ref{eq:3.b} is proven} \\
\end{align*}
With both\ \ref{eq:3.a} and\ \ref{eq:3.b} proven, it holds that the equivalence is true. \hfill $\blacksquare$
\pagebreak

\subsection*{Exercise 4}

Let $\mathbb{N} \times \mathbb{N}$ be the Cartesian product of the set of natural numbers with itself consisting of all ordered pairs $(x1, x2)$, such  that $x1 \in \mathbb{N}$ and $x2 \in \mathbb{N}$. We define a relation on its power set $\mathcal{P} (\mathbb{N} \times \mathbb{N})$ as follows:
\[ \forall A, B \in \mathcal{P}(\mathbb{N} \times \mathbb{N}),\ A \sim B \text{ iff } (A \setminus B) \cup (B \setminus A) = C \text{ and  } C \text{ is a finite set. } \]
Determine whether or not $\sim$ is an equivalence relation and justify your answer by checking each of the three properties in the definition of an equivalence relation.

\subsubsection*{Solution}

To determine if $\sim$ is an equivalence relation, we must check the three properties of an equivalence relation: \textit{Reflexivity}, \textit{symmetry} and \textit{transitivity}.
\begin{align*}
  \intertext{To check \textit{reflexivity}, we must verify $\forall A \in \mathcal{P}(\mathbb{N} \times \mathbb{N}),\ A \sim A = C$, where $C$ is a finite set:}
    & (A \setminus A) \cup (A \setminus A) = C \\
    \rightarrow\ \ & \emptyset \cup \emptyset = C \\
    \rightarrow\ \ & \emptyset = C \\
  \intertext{An empty set contains 0 elements, thus $C$ is finite. So the relation is \textit{reflexive}.}
  \intertext{To check \textit{symmetry}, we now verify $\forall A,B \in \mathcal{P}(\mathbb{N} \times \mathbb{N}),\ A \sim B \Rightarrow (A \setminus B) \cup (B \setminus A) = C$, where $C$ is a finite set:}
    & (A \setminus B) \cup (B \setminus A) = C \\
    \rightarrow\ \ & (A \setminus B) \lor (B \setminus A) = C \tag{Definition of a union} \\
    \rightarrow\ \ & (B \setminus A) \lor (A \setminus B) = C \tag{\#32: Law of commutativity} \\
    \rightarrow\ \ & (B \setminus A) \cup (A \setminus B) = C \tag{Applying definition of a union} \\
  \intertext{Here $B \Rightarrow A$ and $C$ is still finite. Therefore the relation is \textit{symmetric}.}
  \intertext{To check \textit{transitivity}, we finally verify $\forall A,B,C \in \mathcal{P}(\mathbb{N} \times \mathbb{N})$, $A \sim B$ and $B \sim C$ such that: \endgraf\ $A \sim B \Rightarrow (A \setminus B) \cup (B \setminus A) = D$ where $D$ is a finite set, \endgraf\ $B \sim C \Rightarrow (B \setminus C) \cup (C \setminus B) = E$ where $E$ is a finite set, \endgraf\ To show $A \sim C$, which is $(A \setminus C) \cup (C \setminus A) = F$ where $F$ is a finite set:}
    & (A \setminus B) \cup (B \setminus A) \\
    \rightarrow\ \ & (A \cap B^{c}) \cup (B \cap A^{c})                           \tag{Definition of set subtraction} \\
    \rightarrow\ \ & {(A \Rightarrow B)}^{c} \cup {(B \Rightarrow A)}^{c}         \tag{\#20} \\
    \rightarrow\ \ & {((A \Rightarrow B) \cap (B \Rightarrow A))}^{c}             \tag{\#19: De Morgan's Law} \\
    \rightarrow\ \ & {(A \Leftrightarrow B)}^{c}                                  \tag{\#22} \\
    \bullet\ \ & {(A \Leftrightarrow B)}^{c}                                      \tag*{\textcircled{1}} \label{eq:4.1} \\
  \\
    & (B \setminus C) \cup (C \setminus B) \\
    \rightarrow\ \ & (B \cap C^{c}) \cup (C \cap B^{c})                           \tag{Definition of set subtraction} \\
    \rightarrow\ \ & {(B \Rightarrow C)}^{c} \cup {(C \Rightarrow B)}^{c}         \tag{\#20} \\
    \rightarrow\ \ & {((B \Rightarrow C) \cap (C \Rightarrow B))}^{c}             \tag{\#19: De Morgan's Law} \\
  \rightarrow\ \ & {(B \Leftrightarrow C)}^{c}                                    \tag{\#22} \\
    \bullet\ \ & {(B \Leftrightarrow C)}^{c}                                      \tag*{\textcircled{2}} \label{eq:4.2} \\
  \\
    & {(A \Leftrightarrow B)}^{c} \cup {(B \Leftrightarrow C)}^{c}                \tag{\#6: Addition with\ \ref{eq:4.1} and\ \ref{eq:4.2}} \\
    \rightarrow\ \ & {((A \Leftrightarrow B) \cap (B \Leftrightarrow C))}^{c}     \tag{\#18: De Morgan's Law} \\
    \rightarrow\ \ & {(A \Leftrightarrow C)}^{c}                                  \tag{\#17} \\
    \rightarrow\ \ & {((A \Rightarrow C) \cap (C \Rightarrow A))}^{c}             \tag{\#22} \\
    \rightarrow\ \ & {(A \Rightarrow C)}^{c} \cup {(C \Rightarrow A)}^{c}         \tag{\#19: De Morgan's Law} \\
    \rightarrow\ \ & (A \cap C^{c}) \cup (C \cap A^{c})                           \tag{\#20} \\
    \rightarrow\ \ & (A \setminus C) \cup (C \setminus A)                        \tag{Definition of set subtraction} \\
  \intertext{Here we have proven $F$ is a subset of finite sets $D$ and $E$. Thus $F$ is a finite set and \textit{transitivity} of $\sim$ is proven.}
\end{align*}
As the three properties of equivalence have been validated, we can conclude that relation $\sim$ is an equivalence relation.
\\\null\hfill$\blacksquare$
\pagebreak

\end{document}
